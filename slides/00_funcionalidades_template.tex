% !TeX root = ../main.tex
% Input 1: ejemplos de funcionalidades del template

\section{Funcionalidades del template}

\begin{frame}{Funcionalidades disponibles}
\begin{itemize}
  \item Entornos matemáticos: teorema, ejemplo, definición, etc.
  \item Bloque de resultado clave: \texttt{resultbox}.
  \item Progresión en clase con \texttt{\textbackslash pause}.
  \item Inclusión de código con \texttt{frame[fragile]} + \texttt{lstlisting}.
\end{itemize}
\end{frame}

\begin{frame}{Ejemplo de teorema}
\begin{theorem}[Forma general]
Si se cumplen los supuestos del modelo lineal clásico, el estimador MCO es insesgado.
\end{theorem}
\end{frame}

\begin{frame}{Ejemplo de demostración con pausas}
\textit{Primero se usa linealidad de la esperanza.}
\pause

\textit{Luego se aplica \(\E(\varepsilon\mid X)=0\).}
\pause

\textit{Finalmente se concluye insesgamiento.}
\end{frame}

\begin{frame}{Ejemplo de resultado clave}
\begin{resultbox}
Este bloque sirve para destacar conclusiones o mensajes centrales de la clase.
\end{resultbox}
\end{frame}

\begin{frame}[fragile]{Ejemplo de inclusión de código}
\begin{lstlisting}[style=coursecode,language=R,caption={Regresion lineal simple en R}]
# Ejemplo en R: modelo de regresion lineal simple
datos <- data.frame(
  y = c(10, 12, 13, 15, 18),
  x = c(1, 2, 3, 4, 5)
)

ajuste <- lm(y ~ x, data = datos)
summary(ajuste)
\end{lstlisting}
\end{frame}
