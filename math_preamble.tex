% !TeX root = main.tex
% Entornos matem?ticos reutilizables para el curso
\usepackage{mathtools}
\usepackage{bm}
\usepackage{listings}

% Nombres en espa?ol para bloques de teoremas de Beamer
\deftranslation[to=spanish]{Theorem}{Teorema}
\deftranslation[to=spanish]{Lemma}{Lema}
\deftranslation[to=spanish]{Corollary}{Corolario}
\deftranslation[to=spanish]{Definition}{Definici?n}
\deftranslation[to=spanish]{Example}{Ejemplo}
\deftranslation[to=Spanish]{Theorem}{Teorema}
\deftranslation[to=Spanish]{Lemma}{Lema}
\deftranslation[to=Spanish]{Corollary}{Corolario}
\deftranslation[to=Spanish]{Definition}{Definici?n}
\deftranslation[to=Spanish]{Example}{Ejemplo}

% Beamer ya define theorem/lemma/corollary/definition/example.
% Solo definimos los que no existan para evitar conflictos.
\makeatletter
\@ifundefined{proposition}{\newtheorem{proposition}{Proposici?n}[section]}{}
\@ifundefined{remark}{\newtheorem{remark}{Observaci?n}[section]}{}
\makeatother

% Macros de notaci?n de regresi?n
\newcommand{\E}{\mathbb{E}}
\newcommand{\Var}{\operatorname{Var}}
\newcommand{\Cov}{\operatorname{Cov}}
\newcommand{\Corr}{\operatorname{Corr}}
\newcommand{\Normal}{\mathcal{N}}
\newcommand{\R}{\mathbb{R}}

\newcommand{\X}{\mathbf{X}}
\newcommand{\y}{\mathbf{y}}
\newcommand{\x}{\mathbf{x}}
\newcommand{\eps}{\boldsymbol{\varepsilon}}
\newcommand{\betavec}{\boldsymbol{\beta}}
\newcommand{\hatbeta}{\hat{\boldsymbol{\beta}}}

% Estilo de codigo para slides
\definecolor{CodeBg}{RGB}{248,250,252}
\definecolor{CodeFrame}{RGB}{209,213,219}
\definecolor{CodeKeyword}{RGB}{0,84,166}
\definecolor{CodeComment}{RGB}{22,128,81}
\definecolor{CodeString}{RGB}{180,83,9}
\definecolor{CodeNumber}{RGB}{107,114,128}

\lstdefinestyle{coursecode}{
  backgroundcolor=\color{CodeBg},
  frame=single,
  rulecolor=\color{CodeFrame},
  framerule=0.6pt,
  basicstyle=\ttfamily\small,
  keywordstyle=\color{CodeKeyword}\bfseries,
  commentstyle=\color{CodeComment}\itshape,
  stringstyle=\color{CodeString},
  numbers=left,
  numberstyle=\scriptsize\color{CodeNumber},
  numbersep=8pt,
  showstringspaces=false,
  tabsize=2,
  breaklines=true,
  breakatwhitespace=true
}

% Componente reutilizable para destacar resultados clave
\newenvironment{resultbox}[1][Resultado clave]
{\begin{alertblock}{#1}}
{\end{alertblock}}

\newcommand{\ResultBox}[1]{%
\begin{resultbox}
#1
\end{resultbox}
}
